\chapter{Theoretical Convergence}\label{chap:convergence}

The ideas and algorithms were detailed in the previous two chapters and now we want to combine them and show that their combination is an algorithm that converges to an optimal solution.
Note that \cref{alg:ipm} is the same infeasible inexact long-step IPM as in~\cite{Monteiro-ConvergenceAnalysisLongStepInfeasibleIPMs,Avron-FasterRandomizedInfeasibleIPMs} but with a different preconditioner.
Thus, the following proofs are very similar to those in their works with only slight changes.
The essential assumptions in the following proofs are that the algorithms work with exact arithmetic, that solutions of \labelcref{eqn:primal-lp,eqn:dual-lp} exist and that \(\mat{A}\) has full row rank.

\Cref{thm:ipm-convergence} is concerned with the convergence and runtime of \cref{alg:ipm} while \cref{thm:approximate-newton-convergence} shows the correctness and runtime of \cref{alg:newton-direction}.
The two algorithms and theorems can be combined by using \cref{alg:newton-direction} in \cref{line:compute-approx-newton} of \cref{alg:ipm} which yields the following theorem.

\begin{theorem}
Assume that \(\gamma \in (0, 1)\) and \(\sigma \in (0, \frac{4}{5})\) are constant and that the initial point \((\vek{x}^0, \vek{y}^0, \vek{s}^0)\) satisfies \((\vek{x}^0, \vek{s}^0) > (\vek{x}^*, \vek{s}^*)\) for some \((\vek{x}^*, \vek{y}^*, \vek{s}^*) \in \mathcal{F}^*\).
Furthermore, assume that \cref{alg:ipm} uses \cref{alg:newton-direction} with \(\e_{\vek{v}} = \gamma \sigma \mu^k / 4\) to determine the approximate Newton direction in every step.
Then the algorithm terminates successfully using only \(O(???)\) flops with probability at least \(99\%\) for \(\delta = O(n^{-2})\).
\end{theorem}