\chapter{Theoretical Convergence}\label{chap:convergence}

The ideas and algorithms were detailed in the previous two chapters and now we want to combine them and show that their combination is an algorithm that converges to an optimal solution.
Note that \cref{alg:ipm} is the same infeasible inexact long-step IPM as in~\cite{Monteiro-ConvergenceAnalysisLongStepInfeasibleIPMs,Avron-FasterRandomizedInfeasibleIPMs} but with a different preconditioner.
Thus, the following proofs are very similar to those in their works with only slight changes.
The essential assumptions in the following proofs are that the algorithms work with exact arithmetic, that solutions of \labelcref{eqn:primal-lp,eqn:dual-lp} exist and that \(\mat{A}\) has full row rank.

\Cref{thm:ipm-convergence} is concerned with the convergence and runtime of \cref{alg:ipm} while \cref{thm:approximate-newton-convergence} shows the correctness and runtime of \cref{alg:newton-direction}.
The two algorithms and theorems can be combined by using \cref{alg:newton-direction} in \cref{line:compute-approx-newton} of \cref{alg:ipm} which yields the following theorem.

\begin{theorem}
Assume that \(\gamma \in (0, 1)\) and \(\sigma \in (0, \frac{4}{5})\) are constant and that the initial point \((\vek{x}^0, \vek{y}^0, \vek{s}^0)\) satisfies \((\vek{x}^0, \vek{s}^0) > (\vek{x}^*, \vek{s}^*)\) for some \((\vek{x}^*, \vek{y}^*, \vek{s}^*) \in \mathcal{F}^*\).
Furthermore, assume that \cref{alg:ipm} uses \cref{alg:newton-direction} with \(\e_{\vek{v}} = \gamma \sigma \mu^k / 4\) to determine the approximate Newton direction in every step.
Then the algorithm terminates successfully using only \(O(???)\) flops with probability at least \(99\%\) for \(\delta = O(n^{-2})\).
\end{theorem}

Let us now turn to the proofs.
A key observation is that shifting the error term in the Newton system with the perturbation vector \(\vek{v}\) so that \((\hat{\Delta\vek{x}}, \hat{\Delta\vek{y}}, \hat{\Delta\vek{s}})\) satisfies \cref{eqn:approx-newton} ensures that at each iteration the residuals \(\vek{r}^k = (\vek{r}_p^k, \vek{r}_d^k)\) lie on the line segment between \(\vek{r}^0\) and \(\vek{0}\), i.e. \(\vek{r}^k = \eta \vek{r}^0\) for some \(\eta \in [0, 1]\).
Moreover, by the definition of the neighbourhood \(\mathcal{N}_{-\infty}\) this \(\eta\) satisifies \(\eta \leq \frac{\mu^k}{\mu^0} = \frac{{\vek{x}^k}^T \vek{s}^k}{{\vek{x}^0}^T \vek{s}^0}\).
Therefore our iterates satisfy all the conditions of the following lemma.

\begin{lemma}[Lemma 3.2 in~\cite{Monteiro-ConvergenceAnalysisLongStepInfeasibleIPMs}]
  Let \((\vek{x}^0, \vek{y}^0, \vek{s}^0) \in \R^n_{>0} \times \R^m \times \R^n_{>0}\) be a point such that \((\vek{x}^0, \vek{s}^0) \geq (\vek{x}^*, \vek{s}^*)\) for some \((\vek{x}^*, \vek{y}^*, \vek{s}^*) \in \mathcal{F}^*\).
  Then for any point \((\vek{x}, \vek{y}, \vek{s})\) such that the corresponding residuals satisfy \(\vek{r} = \eta \vek{r}^0\) for some \(\eta \in [0, 1]\) and \(\eta \leq \vek{x}^T \vek{s} / {\vek{x}^0}^T \vek{s}^0\) we have \(\eta ({\vek{x}^0}^T \vek{s} + \vek{x}^T {\vek{s}^0}) \leq 3 n \mu\).
\end{lemma}

Let \((\vek{x}, \vek{y}, \vek{s})\) denote an iterate during the course of \cref{alg:ipm}, \((\hat{\Delta\vek{x}}, \hat{\Delta\vek{y}}, \hat{\Delta\vek{s}})\) the Newton direction determined in \cref{line:compute-approx-newton} and 
\begin{align}
  (\vek{x}(\alpha), \vek{y}(\alpha), \vek{s}(\alpha)) &\coloneqq (\vek{x}, \vek{y}, \vek{s}) + \alpha (\hat{\Delta\vek{x}}, \hat{\Delta\vek{y}}, \hat{\Delta\vek{s}}) \\
  \mu(\alpha) &\coloneqq \vek{x}(\alpha)^T \vek{s}(\alpha) \\
  \vek{r}(\alpha) &\coloneqq (\mat{A}\vek{x}(\alpha) - \vek{b}, \mat{A}^T \vek{y}(\alpha) + \vek{s}(\alpha) - \vek{c}).
\end{align}
Using this notation~\citeauthor{Monteiro-ConvergenceAnalysisLongStepInfeasibleIPMs} showed that the stepsize \(\bar{\alpha}\) determined in \cref{alg:ipm} is bounded from below.

\begin{lemma}[Lemma 3.6 in~\cite{Monteiro-ConvergenceAnalysisLongStepInfeasibleIPMs}]
  Assume that \((\hat{\Delta\vek{x}}, \hat{\Delta\vek{y}}, \hat{\Delta\vek{s}})\) satifies \cref{eqn:approx-newton} for some \(\sigma \in (0, \frac{4}{5})\), \((\vek{x}, \vek{y}, \vek{s}) \in \mathcal{N}_{-\infty}(\gamma)\) with \(\gamma \in (0, 1)\) and \(\vek{v} \in \R^n\) satisfying \(\norm{\vek{v}}_\infty \leq \gamma \sigma \mu / 4\).
  Then the stepsize \(\bar{\alpha}\) determined in \cref{line:alpha-tilde,line:alpha-bar} satifies
  \[ \bar{\alpha} \geq \min \Set{1, \frac{\min \Set{\gamma \sigma, 1 - \frac{5}{4}\gamma} \mu}{4 \norm{\hat{\Delta\vek{x}} \circ \hat{\Delta\vek{s}}}_\infty}} \]
  and
  \[ \mu(\bar{\alpha}) \leq \Paren{1 - \Paren{1 - \frac{5}{4}\sigma} \frac{\bar{\alpha}}{2}} \mu. \]
\end{lemma}

Note that the assumptions in this lemma are not stated in this form in the original lemma but are instead inferred from the definition of their algorithm.
In the proof they show that \(\norm{\vek{v}}_\infty \leq \gamma \sigma \mu / 4\) and then deduce the results from this.
It suffices now to upper-bound \(\norm{\hat{\Delta\vek{x}} \circ \hat{\Delta\vek{s}}}_\infty\) to show that \(\mu\) decreases enough in each iteration.

\begin{lemma}[Lemma 16 in~\cite{Avron-FasterRandomizedInfeasibleIPMs}, Lemma 3.7 in~\cite{Monteiro-ConvergenceAnalysisLongStepInfeasibleIPMs}]
  Let \((\vek{x}^0, \vek{y}^0, \vek{s}^0) \in \R^n_{>0} \times \R^m \times \R^n_{>0}\) be a point such that \((\vek{x}^0, \vek{s}^0) \geq (\vek{x}^*, \vek{s}^*)\) for some \((\vek{x}^*, \vek{y}^*, \vek{s}^*) \in \mathcal{F}^*\).
  Let \((\vek{x}, \vek{y}, \vek{s}) \in \mathcal{N}_{-\infty}(\gamma)\) be such that \(\vek{r} = \eta \vek{r}^0\) for some \(\eta \in [0, 1]\) and \(\norm{\vek{v}}_2 \leq \gamma \sigma \mu / 4 \).
  Then \cref{eqn:approx-newton} implies
  \[ \max\Set{\norm{\mat{D}^{-1}\hat{\Delta\vek{x}}}_2, \norm{\mat{D}\hat{\Delta\vek{s}}}_2} \leq \Paren{1 + \frac{\sigma^2}{1- \gamma} - 2\sigma}^{1/2} \sqrt{n \mu} + \frac{6n}{\sqrt{1 - \gamma}} \sqrt{\mu} + \frac{\gamma \sigma}{4 \sqrt{1 - \gamma}}\sqrt{\mu}\]
\end{lemma}

\begin{proof}[Proof of \cref{thm:ipm-convergence}]
    
\end{proof}