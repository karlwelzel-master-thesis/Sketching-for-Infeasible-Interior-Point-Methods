\chapter{Notation}

Vectors are denoted in bold \(\vek{x}\), matrices with bold uppercase letters \(\mat{A}\). For naming primal, dual and slack variables as well as the choice of the neighbourhood we follow \cite{Monteiro-ConvergenceAnalysisLongStepInfeasibleIPMs}.

The standard form linear programming (LP) problem is given by
\begin{equation}
 \min \Set{\vek{c}^T \vek{x} | \mat{A}\vek{x} = \vek{b}, \ \vek{x} \geq 0} 
\end{equation}
and its dual is given by
\begin{equation}
  \max \Set{\vek{b}^T \vek{y} | \mat{A}^T \vek{y} + \vek{s} = \vek{c}, \ \vek{s} \geq 0 }
\end{equation}
where \(\mat{A} \in \R^{m \times n}\), \(\vek{b} \in \R^m\) and \(\vek{c} \in \R^n\) are given and \(\vek{x}, \vek{s} \in \R^n\) and \(\vek{y} \in \R^m\) are the variables.

The necessary and sufficient optimality conditions for both problems are given by
\begin{subequations} \label{optimality-conditions}
  \begin{align}
    \mat{A} \vek{x} &= \vek{b} \label{primal-feasibility} \\
    \mat{A}^T \vek{y} + \vek{s} &= \vek{c} \label{dual-feasibility} \\
    \vek{x} \circ \vek{s} &= 0 \label{complementarity} \\
    \vek{x}, \vek{s} &\geq 0 \label{nonnegativity}
  \end{align}
\end{subequations}

In primal-dual interior-point methods one maintains iterates \((\vek{x}^k, \vek{y}^k, \vek{s}^k)\) that satisfy \(\vek{x}^k, \vek{s}^k > 0\) element-wise.
The \enquote{distance} to the solution is measured using \(\mu = \vek{x}^T \vek{s}/n\) which is equivalent to \(\left(\vek{c}^T\vek{x} - \vek{b}^T\vek{y}\right)/n\) if \(\vek{x}, \vek{y}, \vek{s}\) are feasible.
Feasible IPMs require the iterates to always be feasible (satisfying \cref{primal-feasibility} and \cref{dual-feasibility}) while infeasible IPMs allow some non-zero residuals
\begin{align}
  \vek{r}^k_p &= \mat{A}\vek{x}^k - \vek{b} \\
  \vek{r}^k_d &= \mat{A}^T \vek{y}^k + \vek{s}^k - \vek{c}
\end{align}
as long as their size goes to zero at approximately the same rate as \(\mu\).

The \emph{central path} is given by \(\Set{(\vek{x}_\tau, \vek{y}_\tau, \vek{s}_\tau) | \tau > 0}\) where \(\vek{x}_\tau, \vek{y}_\tau, \vek{s}_\tau\) are uniquely determined by
\begin{subequations}
  \begin{align}
    \mat{A}\vek{x}_\tau - \vek{b} &= \tau \vek{r}_p^0 \\
    \mat{A}^T \vek{y}_\tau + \vek{s}_\tau - \vek{c} &= \tau \vek{r}_d^0 \\
    \vek{x}_\tau \circ \vek{s}_\tau &= \tau \mu_0 \vek{1} \\
    \vek{x}_\tau, \vek{s}_\tau &> 0
  \end{align}
\end{subequations}
if a solution satisfying \cref{optimality-conditions} exists, see \cite{Mizuno-PolynomialTimeConvergenceInexactIPM}.
In this case, the elements of the central path converge towards a solution as \(\tau \to 0\).
In a \emph{long-step} infeasible IPM all iterates are required to belong to a neighbourhoood of this central path given by
\begin{equation}
  \mathcal{N}_{-\infty}(\gamma) = \Set{ (\vek{x}, \vek{y}, \vek{s}) | \vek{x}, \vek{s} > 0, \ \vek{x} \circ \vek{s} \geq (1 - \gamma) \mu \vek{1}, \ \frac{\norm{\vek{r}}}{\norm{\vek{r}^0}} \leq \frac{\mu}{\mu^0} }
\end{equation}
where \(\gamma \in (0, 1)\) is a parameter controlling the size of the neighbourhood and \(\vek{r}^0 = (\vek{r}_p^0, \vek{r}_d^0)\).
If it can be shown that \(\mu \to 0\) during the algorithm then by the choice of the neighbourhood the residuals must also converge to \(0\).

The search direction in every iteration is determined by a Newton system, similar to the one used when trying to find a root of 
\begin{equation}
  F(\vek{x}, \vek{y}, \vek{s}) \deq \begin{pmatrix} \mat{A}\vek{x} - \vek{b} \\ \mat{A}^T \vek{y} + \vek{s} - \vek{c} \\ \vek{x} \circ \vek{s} \end{pmatrix}
\end{equation}
The total derivative of \(F\) is given by
\begin{equation}
  F'(\vek{x}, \vek{y}, \vek{s}) = \begin{pmatrix}
    \mat{A} & 0         & 0       \\
    0       & \mat{A}^T & \mat{I} \\
    \mat{S} & 0         & \mat{X} \\
  \end{pmatrix}
\end{equation}
where \(\mat{X} = \diag(\vek{x})\) and \(\mat{S} = \diag(\vek{s})\).
The actual Newton system determining \((\Delta\vek{x}, \Delta\vek{y}, \Delta\vek{s})\) is slightly modified by adding a centering term \(\sigma \mu \vek{1}\) where \(\sigma \in [0,1]\) which allows for larger steps inside the neighbourhood:
\begin{equation}
  \underbrace{
  \begin{pmatrix}
    \mat{A} & 0         & 0       \\
    0       & \mat{A}^T & \mat{I} \\
    \mat{S} & 0         & \mat{X} \\
  \end{pmatrix}
  }_{F'(\vek{x}, \vek{y}, \vek{s})}
  \begin{pmatrix} \Delta\vek{x} \\ \Delta\vek{y} \\ \Delta\vek{s} \end{pmatrix}
  = -
  \underbrace{
  \begin{pmatrix} \vek{r}_p \\ \vek{r}_d \\ \vek{x} \circ \vek{s} \end{pmatrix}
  }_{F(\vek{x}, \vek{y}, \vek{s})}
  + \begin{pmatrix} 0 \\ 0 \\ \sigma \mu \vek{1} \end{pmatrix}
\end{equation}