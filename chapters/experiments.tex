\chapter{Experiments}\label{chap:experiments}

In this chapter we evaluate the preconditioning technique introduced in \cref{chap:sketching} inside the implementation of the IPM algorithm of \textcite{AndersenAndersen-MosekInteriorPointMethod} provided by the open source Python-based scientific computing library \texttt{scipy}~\cite{Scipy}.

\section{The Homogeneous Algorithm}

The interior-point method implemented in \texttt{scipy} is based on the MOSEK interior-point optimizer as published by \textcite{AndersenAndersen-MosekInteriorPointMethod}.
The most important difference to the general infeasible IPM described in \cref{alg:ipm} is its use of the homogeneous model presented in~\cite{XuHungYe-SimplifiedHomogeneousAlgorithm}.
The primal LP problem in standard form \cref{eqn:primal-lp} is replaced by the self-dual formulation
\begin{subequations}
  \begin{align}
    \mat{A}\vek{x} - \vek{b} \tau &= \vek{0} \\
    \mat{A}^T \vek{y} + \vek{s} - \vek{c}\tau &= \vek{0} \\
    - \vek{c}^T\vek{x} + \vek{b}^T\vek{y} - \kappa &= 0 \\
    \vek{x}, \vek{s} &\geq \vek{0} \\
    \tau, \kappa &\geq 0
  \end{align}
\end{subequations}
where \(\tau\) and \(\kappa\) are additional variables and the objective function is zero.
Observe that setting \(\vek{x}\), \(\vek{y}\), \(\vek{s}\), \(\tau\) and \(\kappa\) to zero gives a solution, so the LP above is always feasible.
The crucial advantage of using this LP formulation is that there is always a solution where \((\vek{x}, \tau)\) and \((\vek{s}, \kappa)\) are strictly complimentary and every such solution can be transformed into either a solution of the original LP (if \(\tau > 0\)) or a certificate that the original LP is infeasible or unbounded (if \(\kappa > 0\)).

% Defining 
% \begin{subequations}
%   \begin{align}
%     \vek{r}_p &\coloneqq \mat{A}\vek{x} - \vek{b} \tau \\
%     \vek{r}_d &\coloneqq \mat{A}^T\vek{y} + \vek{s} - \vek{c}\tau \\
%     \vek{r}_g &\coloneqq -\vek{c}^T \vek{x} + \vek{b}^T \vek{y} - \kappa \\
%     \mu &\coloneqq (\vek{x}^T\vek{s} + \tau \kappa) / (n+1)
%   \end{align}
% \end{subequations}
% the corresponding Newton system that needs to be solved is 
% \begin{equation}
%   \begin{pmatrix}
%     \mat{A}    & -\vek{b} & \mat{0}   & \mat{0} & \mat{0} \\
%     \mat{0}    & -\vek{c} & \mat{A}^T & \mat{I} & \mat{0} \\
%     -\vek{c}^T & \mat{0}  & \vek{b}^T & \mat{0} & -1 \\
%     \mat{S}    & \mat{0}  & \mat{0}   & \mat{X} & \mat{0} \\
%     \mat{0}    & \kappa   & \mat{0}   & \mat{0} & \tau
%   \end{pmatrix}
%   \begin{pmatrix}
%     \Delta\vek{x} \\
%     \Delta\tau \\
%     \Delta\vek{y} \\
%     \Delta\vek{s} \\
%     \Delta\kappa
%   \end{pmatrix}
%   =
%   \begin{pmatrix}
%     -\sigma \vek{r}_p \\
%     -\sigma \vek{r}_d \\
%     -\sigma \vek{r}_g \\
%     -\vek{x} \circ \vek{s} + \sigma \mu \vek{1} \\
%     -\tau \kappa + \sigma \mu
%   \end{pmatrix}
% \end{equation}